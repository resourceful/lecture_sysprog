% Author: Seongjin Lee
% Hanyang University, Seoul, Korea
% esos.hanyang.ac.kr
% 2016-09-11
% note: some slides are adopted from  \url{www.cs.stevens.edu/~jschauma/631A/}
% https://github.com/resourceful/lecture_sysprog/

\documentclass[newPxFont,sthlmFooter,nooffset]{beamer}
\usepackage{kotex}
%\usetheme{sthlm}
\usepackage{../beamer_template/beamerthemesthlm}
\hypersetup{pdfauthor={Makhbayev Mirzhan (maxbaev383@mail.ru)},
            pdfsubject={Lecture Note: System Programming},
            pdfkeywords={Lecture Note, System Programming, class, (under)graduate},
            pdfmoddate={D: \pdfdate},
            pdfcreator={Makhbayev Mirzhan}}

%\setbeamertemplate{footline}[text line]{%
%    \parbox{\linewidth}{\vspace*{-8pt} \insertsectionhead  \hfill\insertshortauthor\hfill\insertpagenumber}}
%\setbeamertemplate{navigation symbols}{}




\title{Topics in Software Systems}
\subtitle{Topic 2: UNIX Standardization and Implementation}
\author{Makhbayev Mirzhan}

\date{2018/09/27}

\begin{document}

\frame[plain]{\titlepage}

\frame{\frametitle{Table of contents}\tableofcontents}


%---------------------------------------------------------

\section{What is UNIX?}

\begin{frame}[t]
  \frametitle{What is UNIX?}
\begin{itemize}
	\item UNIX is an operating system which was first developed in the 1960s, and has been under constant development ever since. By operating system, we mean the suite of programs which make the computer work. It is a stable, multi-user, multi-tasking system for servers, desktops and laptops. 
  \item UNIX systems also have a graphical user interface (GUI) similar to Microsoft Windows which provides an easy to use environment. However, knowledge of UNIX is required for operations which aren't covered by a graphical program, or for when there is no windows interface available, for example, in a telnet session.
  \item There are many different versions of UNIX, although they share common similarities. The most popular varieties of UNIX are Sun Solaris, GNU/Linux, and MacOS X. 
\end{itemize}
    
\end{frame}

\section{UNIX Standardization}
	\begin{frame}[t]
  	\frametitle{UNIX Standardization}
	\begin{itemize}
		\item Standardization or standardisation is the process of implementing and developing technical standards based on the consensus of different parties that include firms, users, interest groups, standards organizations and governments.
		\item The C language has been standardized by the American National Standards Institute (ANSI) since 1989 and thereafter by the International Organization for Standardization (ISO).
	\end{itemize}
	\end{frame}

\begin{frame}[t]
	\frametitle{Standards}
	\begin{itemize}
		\item The standard defines the syntax and semantics of the programming language as well as provide a standard library. This library is important because all contemporary UNIX systems provide the library routines that are specified in the C standard. Here is a list of header files that are included in the standard library.
	\end{itemize}
\end{frame}

\begin{frame}[t]
  \frametitle{Standards calls}
  \begin{itemize}
  \item \texttt{<assert.h>} Conditionally compiled macro that compares its argument to zero
\item \texttt{<complex.h>} Complex number arithmetic
\item \texttt{<ctype.h>} Functions to determine the type contained in character data
\item \texttt{<errno.h>} Macros reporting error conditions
\item \texttt{<fenv.h>} Floating-point environment
\item \texttt{<float.h>} Limits of float types
\item \texttt{<inttypes.h>} Format conversion of integer types
\item \texttt{<iso646.h>} Alternative operator spellings and etc.
  \end{itemize}
\end{frame}

\section{UNIX Implementation}
	\begin{frame}[t]
		\frametitle{Implementation}
		\begin{itemize}
\item The concept of daemons is part and parcel of a Unix implementation. There are some daemons that the system administrator has to know reasonably well. These are the programs that determine what users see when they enter the environment created by the operating system.

\item The first program that runs when a Unix system starts up is the program, INIT. This program sets up the basic machine, executes a special shell script of unix commands that starts any daemons that should be started, and then opens a program called "getty" on each access port to the system.

\item Getty, which is a unix abbreviation for "get teletype" is the daemon which waits at an access port for a query from a user who wishes to login to the system. The characteristics that the port presents to the user are defined in a file called "gettydefs." Notice that in Unix, Login and login are different.
		\end{itemize}
	\end{frame}
\end{document}