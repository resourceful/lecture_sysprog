\documentclass{beamer}
\usetheme{Warsaw}
\usecolortheme{crane}


\title{Files and Directories}
\author{Malikov Davronbek}
\institute{Distributed Systems Lab.Gyeongsang National University}

\date{11.10.2018}

\begin{document}
\begin{frame}

\titlepage 
\end{frame}	
	
	
	
	
	
	
	
  	
\begin{frame}[t]\vspace*{10pt}
\begin{enumerate}
	\item Introduction
	\item stat, fstat, fstatat, and lstat Functions
	\item File Types
	\item Set-User-ID and Set-Group-ID
	\item File access permission
	\item Ownership of New Files and Directories
	\item File Size
	\item link,linkat,unlink,unlinkat, and remove Function
	\item File times
	\item Reading Directories
	\item Summary 

\end{enumerate}
\end{frame}

\begin{frame}[t]{Introduction}\vspace*{4pt}
In the last lesson  we covered the basic functions that perform I/O.The main important notation was I/O for regular files—opening a file, and reading or writing a file.We’ll now look at additional features of the file system and the properties of a file.We’ll start with the stat functions and go through each member of the stat structure,looking at all the attributes of a file. In this process, we’ll also describe each of the functions that modify these attributes: change the owner, change the permissions, and
so on.

\end{frame}
 \begin{frame}[t]{stat, fstat, fstatat, and lstat Functions }\vspace*{4 pt}
 Now We will start with four Functions stat such as , fstat, fstatat, and lstat Functions and information  they return.
 Given a pathname, the stat function returns a structure of information about the named file. The fstat function obtains information about the file that is already open on the descriptor fd. The lstat function is similar to stat, but when the named file is a symbolic link, lstat returns information about the symbolic link, not the file referenced by the symbolic link.


\end{frame}
   \begin{frame}[t]{stat,fstat,fstatat,and Istat Functions}\vspace*{4pt}
    The fstatat function provides a way to return the file statistics for a pathname relative to an open directory represented by the fd argument. The flag argument controls whether symbolic links are followed; when the AT\_SYMLINK\_NOFOLLOW flag is set, fstatat will not follow symbolic links, but rather returns information about the link itself.Otherwise, the default is to follow symbolic links, returning information about the file to which the symbolic link points. If the fd argument has the value
    AT\_FDCWD and the pathname argument is a relative pathname, then fstatat evaluates the pathname argument relative to the current directory.If the pathname argument is an absolute pathname, then ft  argument is ignored. In these two cases, fstatat behaves like either stat or lstat, depending on the value of flag.The buf argument is a pointer to a structure that we must supply. The functions fill in the structure. The definition of the structure can differ among implementations.    
\end{frame}
\begin{frame}[t]{stat,fstat,fstatat,and Istat Functions}
The timespec structure type defines time in terms of seconds and nanoseconds. It
includes at least the following fields:
time\_t tv\_sec;
long tv\_nsec;
Prior to the 2008 version of the standard, the time fields were named st\_atime, st\_mtime,
and st\_ctime, and were of type time\_t (expressed in seconds). The timespec structure
enables higher-resolution timestamps. The old names can be defined in terms of the tv\_sec
members for compatibility. For example, st\_atime can be defined as st\_atim.tv\_sec.
\end{frame}



\begin{frame}[t]{stat,fsat,fstatat,and Istat Functions}\vspace*{4pt}

Note that most members of the stat structure are specified by a primitive system data type.The biggest user of the stat functions is probably the ls -l command, to learn all the information about a file.






\end{frame}
\begin{frame}[t]{File types}\vspace*{4pt}

We’ve talked about two different types of files so far: regular files and directories. Most files on a UNIX system are either regular files or directories.However it does not mean there are only two types of files becouse there are additional  types of files.They types are:
\begin{enumerate}
	\item Regular file.
	\item Directory file.
	\item Block special file.
	\item Character special file.
	\item FIFO.
	\item Socket.
	\item Symbolic link.
\end{enumerate}
 
 
 


\end{frame}

\begin{frame}[t]{File types} \vspace*{4 pt}
\begin{enumerate}
	\item[\textbf{1}] \textbf{Regular file}. The most common type of file, which contains data of some form.	There is no distinction to the UNIX kernel whether this data is text or binary.Any interpretation of the contents of a regular file is left to the application	processing the file.One notable exception to this is with binary executable files. To execute a program, the	kernel must understand its format. All binary executable files conform to a format that allows the kernel to identify where to load a program’s text and data.
	\item[\textbf{2}] \textbf{Directory file}. A file that contains the names of other files and pointers to information on these files. Any process that has read permission for a directory	file can read the contents of the directory, but only the kernel can write directly to a directory file. Processes must use the functions described in this chapter to	make changes to a directory.


\end{enumerate}






\end{frame}
 \begin{frame}[t]{File types}\vspace*{4pt}
\begin{enumerate}
	
	\item[\textbf{3}] \textbf{Block special file. } A type of file providing buffered I/O access in fixed-size units	to devices such as disk drives.\textbf{Note that FreeBSD no longer supports block special files. All access to devices is through the character special interface.}
	\item[\textbf{4}]\textbf{Character special file}. A type of file providing unbuffered I/O access in	variable-sized units to devices. All devices on a system are either block special	files or character special files. 
	\item[\textbf{5}]\textbf{FIFO.} A type of file used for communication between processes. It’s sometimes called a named pipe.
	
	
	
	
	
\end{enumerate}




\end{frame}
  
  
  
 \begin{frame}[t]{File types} \vspace*{4pt}
\begin{enumerate}
	\item[\textbf{6}]\textbf{Socket}A type of file used for network communication between processes. A	socket can also be used for non-network communication between processes on a single host.
	\item[\textbf{7}]\textbf{Symbolic link} A type of file that points to another file. We talk more about
	symbolic links later.
\end{enumerate}




\end{frame}
  
  
  
  
  \begin{frame}[t]{ Set-User-ID and Set-Group-ID}\vspace*{4pt}
  Every process has six or more IDs associated with it.\textbf{They are:}
  \begin{enumerate}
  	
  	\item Real User ID(Who we really are)
  	\item Real Group ID(Who  we really are)
  	\item effective user ID(used for file access permission checks)
  	\item effective group ID(used for file access permission checks)
  	\item supplementry groups ID(used for file access permission checks)
  	\item saved-set user ID(saved by exec functions)
  	\item saved-set group ID(saved by exec functions)
  	
  \end{enumerate}






\end{frame} 
\begin{frame}[t]{Set-user-ID and Set-group-ID}
\begin{enumerate}
	\item The real user ID and real group ID identify who we really are. These two fields
	are taken from our entry in the password file when we log in. Normally, these
	values don’t change during a login session, although there are ways for a
	superuser process to change them
	\item The effective user ID, effective group ID, and supplementary group IDs
	determine our file access permissions
	\item The saved set-user-ID and saved set-group-ID contain copies of the effective
	user ID and the effective group ID, respectively, when a program is executed.
	We describe the function of these two saved values.
\end{enumerate}

\end{frame}
 \begin{frame}[t]{Set-user-ID and Set-group-ID}
 
 Normally, the effective user ID equals the real user ID, and the effective group ID equals
 the real group ID.
 Every file has an owner and a group owner. The owner is specified by the st\_uid
 member of the stat structure; the group owner, by the st\_gid member.
 When we execute a program file, the effective user ID of the process is usually the
 real user ID, and the effective group ID is usually the real group ID. However, we can
 also set a special flag in the file’s mode word (st\_mode) that says, ‘‘When this file is
 executed, set the effective user ID of the process to be the owner of the file (st\_uid).’’
 Similarly, we can set another bit in the file’s mode word that causes the effective group
 ID to be the group owner of the file (st\_gid). These two bits in the file’s mode word
 are called the set-user-ID bit and the set-group-ID bit.
 
 
 
 
 
 
\end{frame}
\begin{frame}[t]{Set-user-ID and Set-group-ID}
For example, if the owner of the file is the superuser and if the file’s set-user-ID bit
is set, then while that program file is running as a process, it has superuser privileges.
This happens regardless of the real user ID of the process that executes the file. As an
example, the UNIX System program that allows anyone to change his or her password,
passwd(1), is a set-user-ID program. This is required so that the program can write the
new password to the password file, typically either /etc/passwd or /etc/shadow,
files that should be writable only by the superuser. Because a process that is running
set-user-ID to some other user usually assumes extra permissions, it must be written
carefully.




\end{frame}
\begin{frame}[t]{Set-user-ID and Set-group-ID}
Returning to the stat function, the set-user-ID bit and the set-group-ID bit are
contained in the file’s st\_mode value. These two bits can be tested against the
constants S\_ISUID and S\_ISGID, respectively.
\end{frame}





\begin{frame}[t]{File access permission}
The st\_mode value also encodes the access permission bits for the file. When we say
file, we mean any of the file types that we described earlier. All the file
types — directories, character special files, and so on—have permissions. Many people
think of only regular files as having access permissions.








\end{frame} 
\begin{frame}[t]{File access directions}

\textbf{1.}The first rule is that whenever we want to open any type of file by name, we must
have execute permission in each directory mentioned in the name, including the
current directory, if it is implied. This is why the execute permission bit for a
directory is often called the search bit.
For example, to open the file /usr/include/stdio.h, we need execute
permission in the directory /, execute permission in the directory /usr, and execute
permission in the directory /usr/include. We then need appropriate permission
for the file itself, depending on how we’re trying to open it: read-only, read–write,
and so on.
If the current directory is /usr/include, then we need execute permission in the
current directory to open the file stdio.h. This is an example of the current
directory being implied, not specifically mentioned. It is identical to our opening the
file ./stdio.h.




\end{frame}
\begin{frame}[t]{File access permissions}
Note that read permission for a directory and execute permission for a directory
mean different things. Read permission lets us read the directory, obtaining a list of
all the filenames in the directory. Execute permission lets us pass through the
directory when it is a component of a pathname that we are trying to access.






\end{frame}
\begin{frame}[t]{File access permissions}
\textbf{2.} The read permission for a file determines whether we can open an existing file for
reading: the O\_RDONLY and O\_RDWR flags for the open function.
\textbf{3.} The write permission for a file determines whether we can open an existing file for
writing: the O\_WRONLY and O\_RDWR flags for the open function.
\textbf{4.} We must have write permission for a file to specify the O\_TRUNC flag in the open
function.
\textbf{5.} We cannot create a new file in a directory unless we have write permission and
execute permission in the directory.
\\textbf{6.} delete an existing file, we need write permission and execute permission in the
directory containing the file. We do not need read permission or write permission
for the file itself.
\textbf{7.} Execute permission for a file must be on if we want to execute the file using any of
the seven exec functions.The file also has to be a regular file.







\end{frame}
\begin{frame}[t]{File access permissions}
The file access tests that the kernel performs each time a process opens, creates, or
deletes a file depend on the owners of the file (st\_uid and st\_gid), the effective IDs
of the process (effective user ID and effective group ID), and the supplementary group
IDs of the process, if supported. The two owner IDs are properties of the file, whereas
the two effective IDs and the supplementary group IDs are properties of the process.
The tests performed by the kernel are as follows:
\textbf{1.} If the effective user ID of the process is 0 (the superuser), access is allowed. This
gives the superuser free rein throughout the entire file system.
\textbf{2.} If the effective user ID of the process equals the owner ID of the file (i.e., the
process owns the file), access is allowed if the appropriate user access
permission bit is set. Otherwise, permission is denied. By appropriate access
permission bit, we mean that if the process is opening the file for reading, the
user-read bit must be on. If the process is opening the file for writing, the
user-write bit must be on. If the process is executing the file, the user-execute bit
must be on.






\end{frame}
\begin{frame}[t]{File access permissions}
\textbf{3.} If the effective group ID of the process or one of the supplementary group IDs of
the process equals the group ID of the file, access is allowed if the appropriate
group access permission bit is set. Otherwise, permission is denied.
\textbf{4.} If the appropriate other access permission bit is set, access is allowed.
Otherwise, permission is denied.
\******* These four steps are tried in sequence. \***\textbf{Note that} if the process owns the file
(step 2), access is granted or denied based only on the user access permissions; the
group permissions are never looked at. Similarly, if the process does not own the file
but belongs to an appropriate group, access is granted or denied based only on the
group access permissions; the other permissions are not looked at. 
 





\end{frame}
\begin{frame}[t]{Ownership of New Files and Directories}

The user ID of a new file is set to the effective user ID of the process. POSIX.1
allows an implementation to choose one of the following options to determine the
group ID of a new file:
\textbf{1.} The group ID of a new file can be the effective group ID of the process.
\textbf{2.} The group ID of a new file can be the group ID of the directory in which the file
is being created.
Using the second option—inheriting the directory’s group ID—assures us that all
files and directories created in that directory will have the same group ID as the
directory. This group ownership of files and directories will then propagate down the
hierarchy from that point. This is used in the Linux directory /var/mail, for example.



\end{frame}
\begin{frame}[t]{File Size}

The st\_size member of the stat structure contains the size of the file in bytes. This
field is meaningful only for regular files, directories, and symbolic links.              
For a regular file, a file size of 0 is allowed. We’ll get an end-of-file indication on the
first read of the file. For a directory, the file size is usually a multiple of a number, such
as 16 or 512. We talk about reading directories in Section 4.22.
For a symbolic link, the file size is the number of bytes in the filename. For
example, in the following case, the file size of 7 is the length of the pathname usr/lib:
lrwxrwxrwx 1 root 7 Sep 25 07:14 lib\--- usr/lib
\textbf{Note that symbolic links do not contain the normal C null byte at the end of the name,
	as the length is always specified by st\_size}
Most contemporary UNIX systems provide the fields s
t\_blksize and
st\_blocks. The first is the preferred block size for I/O for the file, and the latter is the
actual number of 512-byte blocks that are allocated. Recall from last lessons that we
encountered the minimum amount of time required to read a file when we used
st\_blksize for the read operations.



\end{frame}

\begin{frame}[t]{link,linkat,unlink,unlinkat and remove Function}
We can use either the link function or the linkat function to create a link
to an existing file.
We can create a new directory entry, newpath, that references the existing file
existingpath. If the newpath already exists, an error is returned. Only the last component
of the newpath is created. The rest of the path must already exist.
With the linkat function, the existing file is specified by both the efd and
existingpath arguments, and the new pathname is specified by both the nfd and newpath
arguments. By default, if either pathname is relative, it is evaluated relative to the
corresponding file descriptor. If either file descriptor is set to AT\_FDCWD, then the
corresponding pathname, if it is a relative pathname, is evaluated relative to the current
directory. 
\end{frame}
\begin{frame}[t]{link,linkat,unlink,unlinkat and remove Function}
If either pathname is absolute, then the corresponding file descriptor
argument is ignored.
When the existing file is a symbolic link, the flag argument controls whether the
linkat function creates a link to the symbolic link or to the file to which the symbolic
link points. If the AT\_SYMLINK\_FOLLOW flag is set in the flag argument, then a link is
created to the target of the symbolic link. If this flag is clear, then a link is created to the
symbolic link itself.
The creation of the new directory entry and the increment of the link count must be
an atomic operation.Most implementations require that both pathnames be on the same file system,
although POSIX.1 allows an implementation to support linking across file systems. If
an implementation supports the creation of hard links to directories, it is restricted to
only the superuser. This constraint exists because such hard links can cause loops in the
file system, which most utilities that process the file system aren’t capable of handling.





\end{frame}
\begin{frame}[t]{link,linkat,unlink,unlinkat and remove Function}
As mentioned earlier, to unlink a file, we must have write permission and execute
permission in the directory containing the directory entry, as it is the directory entry
that we will be removing. Also, as mentioned in Section 4.10, if the sticky bit is set in
this directory we must have write permission for the directory and meet one of the
following criteria:
\begin{enumerate}
	\item  own the file
	\item own the directory
	\item Have superusert privileges
	Only when the link count reaches 0 can the contents of the file be deleted. One
	other condition prevents the contents of a file from being deleted: as long as some
	process has the file open, its contents will not be deleted. When a file is closed, the
	kernel first checks the count of the number of processes that have the file open. If this
	count has reached 0, the kernel then checks the link count; if it is 0, the file’s contents are
	deleted.
\end{enumerate}



\end{frame}
\begin{frame}[t]{link,linkat,unlink,unlinkat and remove Function}
If the pathname argument is a relative pathname, then the unlinkat function
evaluates the pathname relative to the directory represented by the fd file descriptor
argument. If the fd argument is set to the value AT\_FDCWD, then the pathname is
evaluated relative to the current working directory of the calling process. If the
pathname argument is an absolute pathname, then the fd argument is ignored.
The flag argument gives callers a way to change the default behavior of the
unlinkat function. When the AT\_REMOVEDIR flag is set, then the unlinkat function
can be used to remove a directory, similar to using rmdir. If this flag is clear, then
unlinkat operates like unlink.
\end{frame}
\begin{frame}[t]{File Times}
In stat, fstat, fstatat, and lstat Functions, we discussed how the 2008 version of the Single UNIX Specification
increased the resolution of the time fields in the stat structure from seconds to seconds
plus nanoseconds. The actual resolution stored with each file’s attributes depends on
the file system implementation. For file systems that store timestamps in second
granularity, the nanoseconds fields will be filled with zeros. For file systems that store
timestamps in a resolution higher than seconds, the partial seconds value will be
converted into nanoseconds and returned in the nanoseconds fields.


\end{frame}
\begin{frame}[t]{File Times}

Note the difference between the modification time (st\_mtim) and the changed-status
time (st\_ctim). The modification time indicates when the contents of the file were last
modified. The changed-status time indicates when the i-node of the file was last
modified. In this chapter, we’ve described many operations that affect the i-node
without changing the actual contents of the file: changing the file access permissions,
changing the user ID, changing the number of links, and so on. Because all the
information in the i-node is stored separately from the actual contents of the file, we
need the changed-status time, in addition to the modification time.
\end{frame}
\begin{frame}[t]{File Times}
Note that the system does not maintain the last-access time for an i-node. This is
why the functions access and stat, for example, don’t change any of the three times.
The access time is often used by system administrators to delete files that have not
been accessed for a certain amount of time. The classic example is the removal of files
named a.out or core that haven’t been accessed in the past week. The find(1)
command is often used for this type of operation.
The modification time and the changed-status time can be used to archive only
those files that have had their contents modified or their i-node modified.
The ls command displays or sorts only on one of the three time values. By default,
when invoked with either the -l or the -t option, it uses the modification time of a file.
The -u option causes the ls command to use the access time, and the -c option causes
it to use the changed-status time.

\end{frame}
\begin{frame}[t]{Reading Directories}

Directories can be read by anyone who has access permission to read the directory. But
only the kernel can write to a directory, to preserve file system sanity. Recall from
File access permissions that the write permission bits and execute permission bits for a directory
determine if we can create new files in the directory and remove files from the
directory — they don’t specify if we can write to the directory itself.
The actual format of a directory depends on the UNIX System implementation and
the design of the file system. Earlier systems, such as Version 7, had a simple structure:
each directory entry was 16 bytes, with 14 bytes for the filename and 2 bytes for the
i-node number. When longer filenames were added to 4.2BSD, each entry became
variable length, which means that any program that reads a directory is now system
dependent. To simplify the process of reading a directory, a set of directory routines
were developed and are part of POSIX.1.
\end{frame}
\begin{frame}[t]{Reading Directories}

 Many implementations prevent applications
from using the read function to access the contents of directories, thereby further
isolating applications from the implementation-specific details of directory formats.
\end{frame}
\begin{frame}[t]{Reading Directories}
The fdopendir function first appeared in version 4 of the Single UNIX
Specification. It provides a way to convert an open file descriptor into a DIR structure
for use by the directory handling functions.
The telldir and seekdir functions are not part of the base POSIX.1 standard.
They are included in the XSI option in the Single UNIX Specification, so all conforming
UNIX System implementations are expected to provide them.
Recall our use of several of these functions in the program shown in Figure 1.3, our
bare-bones implementation of the ls command.
The dirent structure defined in <dirent.h> is implementation dependent.
Implementations define the structure to contain at least the following two members:

\end{frame}
\begin{frame}[t]{Reading Directories}
\begin{enumerate}
	\item ino\_t d\_ino; /* i-node number */
	\item char d\_name[]; /* null-terminated filename */
\textbf{The d\_ino entry}	 is not defined by POSIX.1, because it is an implementation feature, but it is
	defined as part of the XSI option in POSIX.1. POSIX.1 defines only the d\_name entry in this
	structure.
\textbf{	Note that} the size of the d\_name entry isn’t specified, but it is guaranteed to hold at
	least NAME\_MAX characters, not including the terminating null byte
	Since the filename is null terminated, however, it doesn’t matter how d\_name is defined
	in the header, because the array size doesn’t indicate the length of the filename.
\end{enumerate}

\end{frame}
\begin{frame}[t]{Summary}
 In the Files and Directories has centered on the stat function. We’ve gone through each member in
the stat structure in detail. This, in turn, led us to examine all the attributes of UNIX
files and directories. We’ve looked at how files and directories might be laid out in a file system, and we’ve seen how to navigate the file system namespace. A thorough
understanding of all the properties of files and directories and all the functions that
operate on them is essential to UNIX programming.

\end{frame}





































  
  
  




 

 

 





















\end{document}